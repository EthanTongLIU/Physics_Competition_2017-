\documentclass[../Paper.tex]{subfiles}

\begin{document}

\section{Background}

With the increase of flight distance and the extension of working time on deep space exploration mission, the tranditional spacecraft which relys on the reaction propulsion will need to carry more fuels and energy, which will largely increase the payload of the spacecraft and thus limit its maximum speed.            

In recent years, solar sail as a new type of spacecraft has attracted much attention. It can be powered by solar light pressure without having to carry a large amount of fuel. Therefore, it has been widely used in the interplanetary navigation and deep space exploration. 

\section{Introduction}	

A spacecraft equipped with solar sails is powered by sunlight instead of rockets or fuels. It can sustain a thrust from the inexhaustible sunlight by sailing out a giant ultra thin sail whose thickness is about only 100 atoms. It flies like a sailing boat in the ocean, and adjusts the direction of the sail by changing the angle of the sail(also named attitude angle). As long as its geometry and the attitude angle are appropriate, it can fly in any direction, including the direction of light source. With the thrust of sunlight, the spacecraft can fly to the edge of the solar system and enter into the interstellar space to complete deep space exploration missions. 

The advantages of solar sail spacecraft is that it does not need to carry a large amount of propellant. Although the solar radiation is very small, but the continuous acceleration from the large, ultrathin mirrors can hardness the faint pressure of the Sun's reflected light to move throught the vacuum sapce and reach a considerable speed which is about 5-10 times faster than traditional spacecraft, and suitable for deep space exploration.            
\\

In this paper, we firstly establish the structure model of the solar sail, and analysis its forces. By comparing the magnitude, we ignore the other forces apart from the Sun and the Sun's gravitational pressure. Then lists the differential equation of the motion by Newton Second Law, and give the expression of acceleration which relate to its position. Finally, we obatained the optimal flight plan of solar sail with the realisation of the shortest transit-time and the maximum payload. 

\section{Problem statement}

The spacecraft can reflect the sunlight by carrying a large, lightweight reflective surface that exerts a pressure equal to twice its energy density, thus gaining impetus. This impetus is related to the acceleration of the spacecraft, and its acceleration is influenced by the attitude of the spacecraft. When the attitude angle $\alpha$ of the spacecraft changes, its trajectory will change accordingly. 
\\

When the distance between Earth and Mars is the closest, a rocket will launch a total mass of 2000 kg (sail plus payload) to escape velocity from Earth to Mars, and the spacecraft must have a relative velocity of no more than 9 km/s to ensure a safe landing close to Mars. Assume that the sail is made of material of mass 7 g/m$^2$, we need to design a flight plan for the solar sail spacecraft in order to find the optimal size of the sail so that to maximum the payload and to reduce transit time.


\end{document}