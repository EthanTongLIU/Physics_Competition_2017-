\documentclass[../Paper.tex]{subfiles}

\begin{document}
\section{Introduction}	

%This is introduction\upcite{1}.

With the increase of flight distance and the extension of working time on deep space exploration mission, the tranditional spacecraft which relys on the reaction propulsion will need to carry more fuel and energy, which largely increases the payload of the spacecraft and thus limits its maximum speed.            

In recent years, solar sail as a new type of spacecraft has attracted much attention. It can be powered by solar pressure without having to carry a large amount of fuel. Therefore, it has been widely used in interplanetary navigation and deep space exploration. 

%The spacecraft gains propulsion from the pressure generated by the reflection of sunlight with the aid of large, ultrathin mirrors. Solar sail propulsion technology is no longer dependent on the consumption of propellant, which can reduce the cost and difficulty of launching spacecraft.

The advantages of solar sail spacecraft is that it does not need to carry a large amount of propellant. Although the solar radiation is very small, but the continuous acceleration from the large, ultrathin mirrors can hardness the faint pressure of the sun's reflected light to move throught the vacuum of sapce and reach a considerable speed which is about 5-10 times faster than traditional spacecraft, suitable for deep space exploration.            

In this paper, we studed the AAAAA of solar sail, gave the differential equation of bbbbb, and designed the ccccc. Finally, we obatained the optimal flight plan of solar sail with the realisation of the shortest transit-time and the maximum payload. 


\end{document}