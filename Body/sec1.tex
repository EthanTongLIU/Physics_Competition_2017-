\documentclass[../Paper.tex]{subfiles}

\begin{document}
\section{Introduction}	

%This is introduction\upcite{1}.

With the increase of flight distance and the extension of working time on deep space exploration mission, the tranditional spacecraft which relys on the reaction propulsion will need to carry more fuel and energy, which largely increases the payload of the spacecraft and thus limits its maximum speed.            

In recent years, solar sail as a new type of spacecraft has attracted much attention. It can be powered by solar pressure without having to carry a large amount of fuel. Therefore, it has been widely used in interplanetary navigation and deep space exploration. 

%The spacecraft gains propulsion from the pressure generated by the reflection of sunlight with the aid of large, ultrathin mirrors. Solar sail propulsion technology is no longer dependent on the consumption of propellant, which can reduce the cost and difficulty of launching spacecraft.

The advantages of solar sail spacecraft is that it does not need to carry a large amount of propellant. Although the solar radiation is very small, but the continuous acceleration from the large, ultrathin mirrors can hardness the faint pressure of the sun's reflected light to move throught the vacuum of sapce and reach a considerable speed which is about 5-10 times faster than traditional spacecraft, suitable for deep space exploration.            
\\

In this paper, we studed the ****** of solar sail, gave the differential equation of *******, and designed the ******. Finally, we obatained the optimal flight plan of solar sail with the realisation of the shortest transit-time and the maximum payload. 

\section{Problem statement}

The spacecraft can reflect the sunlight by carrying a large, lightweight reflective surface that exerts a pressure equal to twice its energy density, thus gaining impetus. This impetus is related to the acceleration of the spacecraft, and its acceleration is influenced by the attitude of the spacecraft. When the attitude angle $\alpha$ of the spacecraft changes, its trajectory will change accordingly. 
\\

When the distance between Earth and Mars is the closest, a rocket will launch a total mass of 2000 kg (sail plus payload) to escape velocity from Earth to Mars, and the spacecraft must have a relative velocity of no more than 9 km/s to ensure a safe landing close to Mars. Assume that the sail is made of material of mass 7 g/m$^2$, we need to design a flight plan for the solar sail spacecraft in order to find the optimal size of the sail so that to maximum the payload and to reduce transit time.


\end{document}