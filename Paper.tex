\documentclass[UTF8,a4paper,12pt,twoside]{article}  

% =======================================
%  --------------- 导言部分 ------------
% =======================================

% ===============================
%  -------- 加载基础宏包 --------
% ===============================

\usepackage{ctex} 
\usepackage{subfiles}
\usepackage{amsmath, amssymb, amsfonts, array, mathrsfs} % <-- 数学宏包, 数学样式
\usepackage{latexsym} % <-- 符号
\usepackage{graphicx} % <-- 颜色和图形, 增强图形支持,文本旋转和缩放
\usepackage{subfigure} % <-- 插入并列子图
\usepackage{xcolor} % <-- 颜色
\usepackage{textcomp} % <-- 特殊字符
\usepackage{float} % <-- 浮动
\usepackage{longtable} % <-- 长表格
\usepackage{multicol, multirow} % <-- 多列, 多行, 分栏
\usepackage{lipsum} % <-- 随机文本
\usepackage{setspace} % <-- 行距
\setlength{\parindent}{0pt} % <-- 段前取消空格
\usepackage{pifont} % <-- 特殊符号
\usepackage{times}\usepackage{mathptmx} % <-- Times New Roman 字体
\usepackage{mathpazo} % <-- mathpazo 字体

\usepackage{esint} % <-- 积分符号
\newcommand{\erint}{ \varoiint } % 二重曲面积分
\newcommand{\ferint}{ \displaystyle\varoiint } % 强制性二重曲面积分

\usepackage{appendix} % <-- 附录
\newcommand{\upcite}[1]{\textsuperscript{\textsuperscript{\cite{#1}}}} % <-- 引用参考文献

\makeatletter % `@' now normal "letter"
\@addtoreset{equation}{section}
\makeatother  % `@' is restored as "non-letter"
\renewcommand\theequation{\oldstylenums{\thesection}%
                    .\oldstylenums{\arabic{equation}}}

% ===============================
%  ----- 代码高亮取消注释 -----
% ===============================
\usepackage{minted} % <-- 代码高亮
\renewcommand{\theFancyVerbLine}{\sffamily
\textcolor[rgb]{0.5,0.5,1.0}{\scriptsize
\oldstylenums{\arabic{FancyVerbLine}}}}

% ===============================
%  ------- 重定义一些名称 -------
% ===============================
\renewcommand\contentsname{Contents} % <-- 重定义目录名称
\renewcommand\listfigurename{插图目录} % <-- 重定义插图目录名称
\renewcommand\listtablename{表格目录} % <-- 重定义表格目录名称
\renewcommand\refname{References} % <-- 重定义参考文献名称
\renewcommand\indexname{Index} % <-- 重定义索引名称
\renewcommand\figurename{\rm Figure} % <-- 重定义图片名称
\renewcommand\tablename{Table} % <-- 重定义表格名称
\renewcommand\abstractname{Abstract} % <-- 重定义摘要名称

% ===============================
%  ----- 以下设置链接样式 -----
% ===============================
\usepackage[colorlinks, linkcolor=black, anchorcolor=blue, citecolor=red]{hyperref}

% ===============================
%  ----- 以下设置图片路径 -----
% ===============================
\graphicspath{{Figures/}}

% ===============================
%  ----- 以下设置页面边距 -----
% ===============================
\usepackage[top=3cm, bottom=2.5cm, left=3cm, right=3cm]{geometry}

% ===============================
%  ----- 以下设置页眉页脚 -----
% ===============================
\usepackage{fancyhdr}
\usepackage{lastpage} % <-- 获取总页数
\pagestyle{fancy} 
\fancyhead[R]{ \bfseries Team 674 } % <-- 右页眉 
%\fancyhead[L]{ \bfseries \leftmark } % <-- 左页眉, 显示对应章标题 
\fancyhead[C]{ \bfseries } 
%\fancyfoot[C]{ 金属结构材料大作业 } % <-- 中页脚, 当前页  of 总页数
%\fancyfoot[R]{  } % <-- 右页脚
%\fancyfoot[L]{  } % <-- 左页脚
\renewcommand{\headrulewidth}{0.4pt} % <-- 上方装饰横线
\renewcommand{\footrulewidth}{0.4pt} % <-- 下方装饰横线
\renewcommand{\footruleskip}{5pt} 

% ========================================
%  --------------- 正文部分 ------------
% ========================================
\begin{document} 

% --------------------------------- 标题页 -----------------------------------
\begin{titlepage}
	\vspace*{20mm}
	\begin{center}
		{\zihao{2} Trajectory Optimization of Interplanetary Rendezvous for Solar Sail Spacecraft  } \\[10mm]
		{\zihao{3} Team 674 } \\[5mm] % \uline{\hfill \kaishu\makebox[70mm]{ ... } \hfill} \\[25mm] % 填写
		{\zihao{3} Problem A } \\[6mm] % \uline{\hfill \kaishu\makebox[70mm]{ 146041B } \hfill} \\[25mm] % 填写
		{\bfseries \zihao{3} Abstract } \\[3mm]
	\end{center}
<<<<<<< HEAD
	In this paper, we discusse the feasibility of launching a solar sail spacecraft from Earth to Mars and also optimize the flight plan. In order to save energy, we use the rocket to accelerate the solar sail spacecraft so that can just meet the spacecraft escape range of earth's gravity, thus we consider the transmitter initial velocity and the earth revolution speed as equal. By solving the kinematic differential equations with the aid of Matlab, we calculate and analyse the impact of different trajectories of solar sail area and the attitude angle of the spacecraft, and then compared to the Mars' revolution trajectory, we use Matlab to realize the optimization of the trajectory with the loop nesting and iterative method, and the shortest time is used to finally find out the optimal solution. Get some couples (A, $\alpha$) satisfy the end conditions. Then use the constraint shortest time to find out the optimal solution.  With this approach finally we obtain the minimum transit time $t_{arrive}$is 483 days, and the corrosponding effective load is 387.2 kg account for 19.36$\%$of the total mass. 
=======
	In order to design the optimal trajectory of a solar sail spacecraft launched from Earth to Mars, in this paper, we discuss the influence of the surface area of solar sail and its attitude angle about the largest-payload and shortest-time problem. We treat this influencial problem as an optimal control problem.

will affect the sail area and the attitude angle as endpoint time equation of state optimal constrained time and relative distance inequality the constraint control problem, designed by Matlab iterative algorithm, linear search the optimal attitude angle and the best turn sail area, shorten the time of flight tasks for the engineering application of solar sail, it has practical reference value. 
>>>>>>> bb23a93c301e11ec2ac1cba24dd30e782642402d
\end{titlepage}

% ---------------------------------- 目录 ------------------------------------
\tableofcontents % <-- 目录

% ---------------------------------- 正文 ------------------------------------
\newpage
\begin{spacing}{1.5} % <-- 行距
\subfile{Body/sec1.tex}
\subfile{Body/sec2.tex}
%\subfile{Body/sec3.tex}
%\subfile{Body/sec4.tex}
\subfile{Body/sec5.tex}
\subfile{Body/sec6.tex}

% -------------------------------- 参考文献 -----------------------------------
\newpage
\subfile{Body/reference.tex}

% -------------------------------- 附录(代码) -----------------------------------
\newpage
\subfile{Body/appendix.tex}

\end{spacing}
\end{document}